
\documentclass[12pt,a4paper]{article}

\usepackage{url}
\usepackage{parskip} 	

\RequirePackage{color}
\RequirePackage{graphicx}
\usepackage[usenames,dvipsnames]{xcolor}
\usepackage[scale=0.9]{geometry}

\usepackage{tabularx}

\usepackage{enumitem}

\newcolumntype{C}{>{\centering\arraybackslash}X} 

\usepackage{supertabular}
\usepackage{tabularx}
\newlength{\fullcollw}
\setlength{\fullcollw}{0.47\textwidth}

\usepackage{titlesec}				
\usepackage{multicol}
\usepackage{multirow}

\titleformat{\section}{\Large\scshape\raggedright}{}{0em}{}[\titlerule]
\titlespacing{\section}{0pt}{6pt}{10pt}

\usepackage[style=authoryear,sorting=ynt, maxbibnames=2]{biblatex}

\usepackage[unicode, draft=false]{hyperref}
\definecolor{linkcolour}{rgb}{0,0.2,0.6}
\hypersetup{colorlinks,breaklinks,urlcolor=linkcolour,linkcolor=linkcolour}
\addbibresource{citations.bib}
\setlength\bibitemsep{1em}

\usepackage{fontawesome5}

\begin{document}

\pagestyle{empty} 

 \begin{tabularx}{\linewidth}{@{} C @{}}
\Huge{Noah Yacowar} \\[7.5pt]
\href{https://github.com/noah-yacowar}{\raisebox{-0.05\height}\faGithub\ noah-yacowar} \ $|$ \ 
\href{https://linkedin.com/in/noah-yacowar-0314011b5}{\raisebox{-0.05\height}\faLinkedin\ noah-yacowar} \ $|$ \ 
\href{mailto:nyacowar@uwaterloo.ca}{\raisebox{-0.05\height}\faEnvelope \ nyacowar@uwaterloo.ca} \ $|$ \ 
\href{tel:+16478706663}{\raisebox{-0.05\height}\faMobile \ +1 647-870-6663} \\
\end{tabularx}

\section{Skills} 
\textbf{Languages:} Python, Java, C, C++, C\#, HTML, CSS, JavaScript, LaTeX \\ 
\textbf{Technologies/Frameworks:} Git, Arduino, OpenCV, PyTorch, Jupyter Notebook, MonoGame \\
\textbf{Software:} Unity, Unreal Engine, Github, Visual Studio Code, Microsoft Office

\section{Projects}

\begin{tabularx}{\linewidth}{ @{}l r@{} }
\textbf{Unity First-Person Shooter} \href{https://github.com/NoahYacowar/Unity-HorrorFPS}{\raisebox{-0.05\height}\faLink} & \hfill July 2022 - August 2022 \\[3.75pt]
\multicolumn{2}{@{}X@{}}{
\begin{minipage}[t]{\linewidth}
    \begin{itemize}[nosep,after=\strut, leftmargin=1.45em, itemsep=4pt]
        \item Built horror-themed first-person shooter in unity, in which levels are randomly generated using prefabs.
        \item Implemented \textbf{A* path-finding algorithm} for the enemy, and used a parallel combination of a linked list and Grid System for level generation.
    \end{itemize}
    \end{minipage}
}
\end{tabularx}

\begin{tabularx}{\linewidth}{ @{}l r@{} }
\textbf{Guitar Playing Robot}  \href{https://github.com/NoahYacowar/Guitar-Playing-Robot}{\raisebox{-0.05\height}\faLink} & \hfill September 2022 - November 2022 \\[3.75pt]
\multicolumn{2}{@{}X@{}}{
\begin{minipage}[t]{\linewidth}
    \begin{itemize}[nosep,after=\strut, leftmargin=1.45em, itemsep=4pt]
        \item Built a robot that plays a string on guitar. The system has fretting and strumming mechanisms. 
        \item Encoded notes on coloured paper strips. Used \textbf{motor encoders} to coordinate both mechanisms.
        \item Processed input from a colour and ultrasonic sensor to interpret song notes and timing (\textbf{95\%} accurate).
        \item Programmed functions in \textbf{C} to control onboard mechanisms using the RobotC library. 
    \end{itemize}
    \end{minipage}
}
\end{tabularx}

\begin{tabularx}{\linewidth}{ @{}l r@{} }
\textbf{Personal Website} \href{https://noah-yacowar.github.io/Personal-Website/}{\raisebox{-0.05\height}\faLink} & \hfill December 2022 \\[3.75pt]
\multicolumn{2}{@{}X@{}}{
\begin{minipage}[t]{\linewidth}
    \begin{itemize}[nosep,after=\strut, leftmargin=1.45em, itemsep=4pt]
        \item Produced a fully responsive personal website using HTML, CSS and Vanilla \textbf{Javascript}.
        \item Implemented adaptive website functionality for varying screen sizes using \textbf{@media} rules. 
    \end{itemize}
    \end{minipage}
}
\\[30px]
\end{tabularx}
\begin{tabularx}{\linewidth}{ @{}l r@{} }
\textbf{Toyota Innovation Challenge} & \hfill November 2022 \\[3.75pt]
\multicolumn{2}{@{}X@{}}{
\begin{minipage}[t]{\linewidth}
    \begin{itemize}[nosep,after=\strut, leftmargin=1.45em, itemsep=4pt]
        \item Partook in hackathon using computer vision to track car's wheel to take a picture at appropriate time. 
        \item Built minimum bounding box algorithm which isolated the front wheel with \textbf{±1mm} error.
        \item Built program using \textbf{OpenCV} code snippets, implementing Hough transform and contour detection functions. When the car passes a certain point, a picture is taken. %update
    \end{itemize}
    \end{minipage}
}
\end{tabularx} 

%Experience
\section{Work Experience}
\begin{tabularx}{\linewidth}{ @{}l r@{} }
\textbf{Computer Vision Sub-Team Member} & \hfill December 2022 - present \\
Waterloo Aerial Robotics Group \\[3.75pt]
\multicolumn{2}{@{}X@{}}{
\begin{minipage}[t]{\linewidth}
    \begin{itemize}[nosep,after=\strut, leftmargin=1.45em, itemsep=4pt]
        \item Developed an image classifier trained on the CIFAR-10 dataset using the \textbf{PyTorch} framework.
        \item Used a trained convolutional neural network to interpret test images with \textbf{82\%} accuracy.
        \item Graphed losses and accuracy of the model using \textbf{Matplotlib}.
    \end{itemize}
    \end{minipage}
}
\end{tabularx}


\begin{tabularx}{\linewidth}{ @{}l r@{} }
\textbf{Computer Science Club Founder} & \hfill April 2021 - June 2022 \\
Personal Venture \\[3.75pt]
\multicolumn{2}{@{}X@{}}{
\begin{minipage}[t]{\linewidth}
    \begin{itemize}[nosep,after=\strut, leftmargin=1.45em, itemsep=4pt]
        \item Led group of 7 students alongside 2 other instructors. Taught coding concepts from object-oriented programming to \textbf{data sorting} and \textbf{game development}, meeting 2-3 times per week.
        \item Organized meetings with industry workers to give insight into industry tools and experiences.
    \end{itemize}
    \end{minipage}
}
\end{tabularx}

\section{Education}
\begin{tabularx}{\linewidth}{@{}l X@{}}
\textbf{Candidate for BASc in Mechatronics Engineering} & \hfill September 2022 - present \\
University of Waterloo - President's Scholarship of Distinction

\end{tabularx}

\vfill

\end{document}
